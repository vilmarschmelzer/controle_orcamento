\documentclass[a4paper]{article}
\usepackage[margin=1cm]{geometry}
\usepackage[T1]{fontenc}
\usepackage[utf8x]{inputenc}
\usepackage[brazil]{babel}
\usepackage{amsmath}
\usepackage{amsfonts}
\usepackage{amssymb}
\usepackage{fancyhdr}
\renewcommand{\rmdefault}{phv} % Arial
\renewcommand{\sfdefault}{phv} % Arial


\head{
\begin{table}
\begin{tabular}{l p{11cm} p{3cm}}
Orçamento: {{orcamento.id}} &  & Data: {{ orcamento.dt_criacao|date:"d/m/Y"  }} &
CNPJ/CPF: {{ orcamento.cliente.documento }} & Nome: {{ orcamento.cliente.nome }} & &

	{{contato.tipo_contato}}: {{contato.contato}} & & &

\end{tabular}
\end{table}}

\begin{document}
\begin{table}
\begin{tabular}{c|p{12cm}|r|r|r}                    					  \hline
Código       & Nome       			& Valor unit. 	& Qtd. & Desconto\\\hline


	{{ item.produto.id }} & {{item.produto.nome}} & {{item.vl_unitario}}  & {{item.quantidade}} & {{item.vl_desconto}}\\
	& Obs: {{item.observacao}} & & & \\ \hline

\end{tabular}
\end{table}}
\begin{flushright}
Total produtos: {{orcamento.get_total_produtos}}\\
Total desconto: {{orcamento.get_total_desconto}}\\
Total: {{orcamento.get_total}}
\end{flushright}
\end{document}
